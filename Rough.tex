\documentclass[a4paper]{article}
\usepackage{graphicx} 
\usepackage[paperwidth=9 in]{geometry} 
\usepackage{amsmath}
\usepackage{slashed}
\usepackage{float}
\usepackage{fancyhdr}
\usepackage{color}
\usepackage{amsfonts}%
\usepackage{amssymb}%
\usepackage{listings}
\usepackage{color}
\usepackage{feynmp-auto}
\DeclareGraphicsRule{*}{mps}{*}{} % for being able to read the produced file
\definecolor{mygreen}{rgb}{0,0.6,0}
\definecolor{mygray}{rgb}{0.5,0.5,0.5}
\definecolor{mymauve}{rgb}{0.58,0,0.82}
\lstset{ %
  backgroundcolor=\color{white},   % choose the background color; you must add \usepackage{color} or \usepackage{xcolor}
  basicstyle=\footnotesize,        % the size of the fonts that are used for the code
  breakatwhitespace=false,         % sets if automatic breaks should only happen at whitespace
  breaklines=true,                 % sets automatic line breaking
  captionpos=b,                    % sets the caption-position to bottom
  commentstyle=\color{mygreen},    % comment style
  deletekeywords={...},            % if you want to delete keywords from the given language
  escapeinside={\%*}{*)},          % if you want to add LaTeX within your code
  extendedchars=true,              % lets you use non-ASCII characters; for 8-bits encodings only, does not work with UTF-8
  frame=single,                    % adds a frame around the code
  keepspaces=true,                 % keeps spaces in text, useful for keeping indentation of code (possibly needs columns=flexible)
  keywordstyle=\color{blue},       % keyword style
  language=Octave,                 % the language of the code
  morekeywords={*,...},            % if you want to add more keywords to the set
  numbers=left,                    % where to put the line-numbers; possible values are (none, left, right)
  numbersep=5pt,                   % how far the line-numbers are from the code
  numberstyle=\tiny\color{mygray}, % the style that is used for the line-numbers
  rulecolor=\color{black},         % if not set, the frame-color may be changed on line-breaks within not-black text (e.g. comments (green here))
  showspaces=false,                % show spaces everywhere adding particular underscores; it overrides 'showstringspaces'
  showstringspaces=false,          % underline spaces within strings only
  showtabs=false,                  % show tabs within strings adding particular underscores
  stepnumber=2,                    % the step between two line-numbers. If it's 1, each line will be numbered
  stringstyle=\color{mymauve},     % string literal style
  tabsize=2,                       % sets default tabsize to 2 spaces
  title=\lstname                   % show the filename of files included with \lstinputlisting; also try caption instead of title
}
\usepackage{tcolorbox}
\usepackage[framemethod=TikZ]{mdframed}
\graphicspath{{figures/}} % Location of the graphics files
\usepackage{booktabs} % Top and bottom rules for table
\usepackage[font=small,labelfont=bf]{caption} % Required for specifying captions to tables and figures
\usepackage{amsfonts, amsmath, amsthm, amssymb} % For math fonts, symbols and environments
\usepackage{wrapfig} % Allows wrapping text around tables and figures

\mdfdefinestyle{MyFrame}{%
    linecolor= white,
    outerlinewidth=2pt,
    roundcorner=20pt,
    innertopmargin=\baselineskip,
    innerbottommargin=\baselineskip,
    innerrightmargin=20pt,
    innerleftmargin=20pt,
    backgroundcolor=gray!20!white}
		
\newtheorem{theorem}{Theorem}
\newtheorem{claim}{Claim}
\newtheorem{Proofs}{Proof}
\newtheorem{definition}{Definition}
\newtheorem{remark}{Remark}


\begin{document}

\pagestyle{fancy}

\numberwithin{equation}{section}

\title{Notes} % Title
\author{D.D \textsc{Kodwani}} % Author name
\date{\today} % Specify a date for the report
\begin{document}
\maketitle % Insert the title, author and date



\setlength\parindent{0pt} % Removes all indentation from paragraph
\renewcommand{\labelenumi}{\alph{}.} % Make numbering in the enumerate environment by letter rather than number (e.g. section 6)

\begin{document}

\section{Intro}

The Fisher info kernel we are interested in is

\begin{equation}
	I (\ln k, \ln k') = \frac{\delta^2( - \ln L)}{\delta P(\ln k) \delta P(\ln k')}
\end{equation}

The current version for the fisher info in the likli.py code 

\begin{equation}
	\delta^2 (-\ln L)_{k,k'} = \sum_l \frac{2l +1}{2} f_s \left( \delta^2 C^T_{kk'} \left( - \frac{C^O}{(C^T + N)^2} - \frac{1}{C^T + N} \right) + \delta C^T_k \delta C^T_{k'} \left( \frac{2 C^O}{(C^T + N)^3} + \frac{1}{(C^T + N)^2} \right) \right)
\end{equation}

where the subscript of $k,k'$ represent the derivatives wrt to that $k$. The variation in the $Cl_s$ is taken to be of the form:

\begin{eqnarray}
	\delta C_k & = & C_l(k) - C_l(ML)	\nonumber	\\
	\delta C_{kk'} & = & 2(C_l(ML))^2 - (C_l(k) - C_l(k'))^2
\end{eqnarray}

where $ML$ stands for Maximum liklihood which we take to be the fiducial model. These forms of derivatives are very Heuristic and need justification. The form for $\delta C_{kk'}$ in particular since the only real condition in deriving it was that it should be symmetric in $C_k$ and $C_{k'}$.  Here $C^T$ with no index is taken to be the Cls from CAMB with the normal/fiducial power spectrum. 

	
	
	
	
	
	
	
	
	

\end{document}